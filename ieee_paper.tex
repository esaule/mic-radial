\documentclass[10pt, conference, compsocconf]{IEEEtran}
\usepackage[usenames,dvipsnames]{color}
%\usepackage[english]{babel}
\usepackage{tabularx}
\usepackage{soul}
\usepackage{xparse}
\usepackage{listings}
%\usepackage[normalem]{ulem}



%%%%%%%%%%%%%%%
% Show a list of items "todo" or "done" 
% USAGE: 
% \begin{todolist} 
% 	\todo Something not finished
% 	\done Something finished
% \end{todolist} 
\newenvironment{todolist}{%
  \begin{list}{}{}% whatever you want the list to be
  \let\olditem\item
  \renewcommand\item{\olditem \textcolor{red}{(TODO)}: }
  \newcommand\todo{\olditem \textcolor{red}{(TODO)}: }
   \newcommand\done{\olditem \textcolor{ForestGreen}{(DONE)}: }
}{%
  \end{list}
} 
%%%%%%%%%%%%%%%

%%%%%%%%%%%%%%%
% Show a Author's Note
% USAGE: 
% \incomplete[Optional footnote message to further clarify note]{The text which is currently not finished}
\DeclareDocumentCommand \incomplete{ o m }
{%
\IfNoValueTF {#1}
{\textcolor{red}{Incomplete: \ul{#2}}} 
{\textcolor{red}{Incomplete: \ul{#2}}\footnote{Comment: #1}}%
}
%%%%%%%%%%%%%%%



%%%%%%%%%%%%%%%
% Show a Author's Note
% USAGE: 
% \authnote[Optional footnote message to further clarify note]{The note to your readers}
\DeclareDocumentCommand \authnote { o m }
{%
\IfNoValueTF {#1}
{\textcolor{blue}{Author's Note: \ul{#2}}} 
{\textcolor{blue}{Author's Note: \ul{#2}}\footnote{Comment: #1}}%
}
%%%%%%%%%%%%%%%



%%%%%%%%%%%%%%%
% Strike out text that doesn't belong in the paper
% USAGE: 
% \strike[Optional footnote to state why it doesn't belong]{Text to strike out}
\DeclareDocumentCommand \strike { o m }
{%
\setstcolor{Red}
\IfNoValueTF {#1}
{\textcolor{Gray}{\st{#2}}} 
{\textcolor{Gray}{\st{#2}}\footnote{Comment: #1}}%
}
%%%%%%%%%%%%%%%

\definecolor{light-gray}{gray}{0.95}

\newcommand{\cbox}[3]{
\ \\
\fcolorbox{#1}{#2}{
\parbox{\textwidth}{
#3
}
}
}

% Setup an environment similar to verbatim but which will highlight any bash commands we have
\lstnewenvironment{unixcmds}[0]
{
%\lstset{language=bash,frame=shadowbox,rulesepcolor=\color{blue}}
\lstset{ %
language=sh,		% Language
basicstyle=\ttfamily,
backgroundcolor=\color{light-gray}, 
rulecolor=\color{blue},
%frame=tb, 
columns=fullflexible,
%framexrightmargin=-.2\textwidth,
linewidth=0.8\textwidth,
breaklines=true,
%prebreak=/, 
  prebreak = \raisebox{0ex}[0ex][0ex]{\ensuremath{\hookleftarrow}},
%basicstyle=\footnotesize,       % the size of the fonts that are used for the code
%numbers=left,                   % where to put the line-numbers
%numberstyle=\footnotesize,      % the size of the fonts that are used for the line-numbers
%stepnumber=2,                   % the step between two line-numbers. If it's 1 each line 
                                % will be numbered
%numbersep=5pt,                  % how far the line-numbers are from the code
showspaces=false,               % show spaces adding particular underscores
showstringspaces=false,         % underline spaces within strings
showtabs=false,                 % show tabs within strings adding particular underscores
frame=single,	                % adds a frame around the code
tabsize=2,	                % sets default tabsize to 2 spaces
captionpos=b,                   % sets the caption-position to bottom
breakatwhitespace=false,        % sets if automatic breaks should only happen at whitespace
}
} { }

% Setup an environment similar to verbatim but which will highlight any bash commands we have
\lstnewenvironment{cppcode}[1]
{
%\lstset{language=bash,frame=shadowbox,rulesepcolor=\color{blue}}
\lstset{ %
	backgroundcolor=\color{light-gray}, 
	rulecolor=\color[rgb]{0.133,0.545,0.133},
	tabsize=4,
	language=[GNU]C++,
%	basicstyle=\ttfamily,
        basicstyle=\scriptsize,
        upquote=true,
        aboveskip={1.5\baselineskip},
        columns=fullflexible,
        %framexrightmargin=-.1\textwidth,
       %framexleftmargin=6mm,
        showstringspaces=false,
        extendedchars=true,
        breaklines=true,
        prebreak = \raisebox{0ex}[0ex][0ex]{\ensuremath{\hookleftarrow}},
        frame=single,
        showtabs=false,
        showspaces=false,
        showstringspaces=false,
        numbers=left,                   % where to put the line-numbers
	numberstyle=\footnotesize,      % the size of the fonts that are used for the line-numbers
	stepnumber=4,                   % the step between two line-numbers. If it's 1 each line 
                                % will be numbered
	firstnumber=#1,
         numbersep=5pt,                  % how far the line-numbers are from the code
        identifierstyle=\ttfamily,
        keywordstyle=\color[rgb]{0,0,1},
        commentstyle=\color[rgb]{0.133,0.545,0.133},
        stringstyle=\color[rgb]{0.627,0.126,0.941},
}
} { }

% Setup an environment similar to verbatim but which will highlight any bash commands we have
\lstnewenvironment{mcode}[1]
{
\lstset{ %
	backgroundcolor=\color{light-gray}, 
	rulecolor=\color[rgb]{0.133,0.545,0.133},
	tabsize=4,
	language=Matlab,
%	basicstyle=\ttfamily,
        basicstyle=\scriptsize,
        upquote=true,
        aboveskip={1.5\baselineskip},
        columns=fullflexible,
        %framexrightmargin=-.1\textwidth,
       %framexleftmargin=6mm,
        showstringspaces=false,
        extendedchars=true,
        breaklines=true,
        prebreak = \raisebox{0ex}[0ex][0ex]{\ensuremath{\hookleftarrow}},
        frame=single,
        showtabs=false,
        showspaces=false,
        showstringspaces=false,
        numbers=left,                   % where to put the line-numbers
	numberstyle=\footnotesize,      % the size of the fonts that are used for the line-numbers
	stepnumber=4,                   % the step between two line-numbers. If it's 1 each line 
                                % will be numbered
	firstnumber=#1,
         numbersep=5pt,                  % how far the line-numbers are from the code
        identifierstyle=\ttfamily,
        keywordstyle=\color[rgb]{0,0,1},
        commentstyle=\color[rgb]{0.133,0.545,0.133},
        stringstyle=\color[rgb]{0.627,0.126,0.941},
}
} { }

\newcommand{\inputmcode}[1]{%
\lstset{ %
	backgroundcolor=\color{light-gray},  %
	rulecolor=\color[rgb]{0.133,0.545,0.133}, %
	tabsize=4, %
	language=Matlab, %
%	basicstyle=\ttfamily,
        basicstyle=\scriptsize, %
        %        upquote=true,
        aboveskip={1.5\baselineskip}, %
        columns=fullflexible, %
        %framexrightmargin=-.1\textwidth,
       %framexleftmargin=6mm,
        showstringspaces=false, %
        extendedchars=true, %
        breaklines=true, %
        prebreak = \raisebox{0ex}[0ex][0ex]{\ensuremath{\hookleftarrow}}, %
        frame=single, %
        showtabs=false, %
        showspaces=false, %
        showstringspaces=false,%
        numbers=left,                   % where to put the line-numbers
	numberstyle=\footnotesize,      % the size of the fonts that are used for the line-numbers
	stepnumber=4,                   % the step between two line-numbers. If it's 1 each line 
                                % will be numbered
         numbersep=5pt,                  % how far the line-numbers are from the code
        identifierstyle=\ttfamily, %
        keywordstyle=\color[rgb]{0,0,1}, %
        commentstyle=\color[rgb]{0.133,0.545,0.133}, %
        stringstyle=\color[rgb]{0.627,0.126,0.941} %
}
\lstinputlisting{#1}%
}

%\lstset{ %
%	backgroundcolor=\color{light-gray}, 
%	rulecolor=\color[rgb]{0.133,0.545,0.133},
%	tabsize=4,
%	language=Matlab,
%%	basicstyle=\ttfamily,
%        basicstyle=\scriptsize,
%        upquote=true,
%        aboveskip={1.5\baselineskip},
%        columns=fullflexible,
%        %framexrightmargin=-.1\textwidth,
%       %framexleftmargin=6mm,
%        showstringspaces=false,
%        extendedchars=true,
%        breaklines=true,
%        prebreak = \raisebox{0ex}[0ex][0ex]{\ensuremath{\hookleftarrow}},
%        frame=single,
%        showtabs=false,
%        showspaces=false,
%        showstringspaces=false,
%        numbers=left,                   % where to put the line-numbers
%	numberstyle=\footnotesize,      % the size of the fonts that are used for the line-numbers
%	stepnumber=4,                   % the step between two line-numbers. If it's 1 each line 
%                                % will be numbered
%	firstnumber=#1,
%         numbersep=5pt,                  % how far the line-numbers are from the code
%        identifierstyle=\ttfamily,
%        keywordstyle=\color[rgb]{0,0,1},
%        commentstyle=\color[rgb]{0.133,0.545,0.133},
%        stringstyle=\color[rgb]{0.627,0.126,0.941},
%}


\newcommand{\Laplacian}[1]{\nabla^2 #1}

% set of all nodes received and contained on GPU
\newcommand{\setAllNodes}[0]{\mathcal{G}}
% set of stencil centers on GPU
\newcommand{\setCenters}[0]{\mathcal{Q}}
% set of stencil centers with nodes in \setDepend
\newcommand{\setBoundary}[0]{\mathcal{B}}
% set of nodes received by other GPUs
\newcommand{\setDepend}[0]{\mathcal{R}}
% set of nodes sent to other GPUs
\newcommand{\setProvide}[0]{\mathcal{O}}


\newcommand{\toprule}[0]{\hline}
\newcommand{\midrule}[0]{\hline\hline}
\newcommand{\bottomrule}[0]{\hline}

\newcolumntype{C}{>{\centering\arraybackslash}b{1in}}
\newcolumntype{L}{>{\flushleft\arraybackslash}b{1.5in}}
\newcolumntype{R}{>{\flushright\arraybackslash}b{1.5in}}
\newcolumntype{D}{>{\flushright\arraybackslash}b{2.0in}}
\newcolumntype{E}{>{\flushright\arraybackslash}b{1.0in}}

\DeclareSymbolFont{AMSb}{U}{msb}{m}{n}
\DeclareMathSymbol{\N}{\mathbin}{AMSb}{"4E}
\DeclareMathSymbol{\Z}{\mathbin}{AMSb}{"5A}
\DeclareMathSymbol{\R}{\mathbin}{AMSb}{"52}
\DeclareMathSymbol{\Q}{\mathbin}{AMSb}{"51}
\DeclareMathSymbol{\PP}{\mathbin}{AMSb}{"50}
\DeclareMathSymbol{\I}{\mathbin}{AMSb}{"49}
%\DeclareMathSymbol{\C}{\mathbin}{AMSb}{"43}

%%%%%% VECTOR NORM: %%%%%%%
\newcommand{\vectornorm}[1]{\left|\left|#1\right|\right|}
\newcommand{\vnorm}[1]{\left|\left|#1\right|\right|}
\newcommand{\by}[0]{\times}
\newcommand{\vect}[1]{\mathbf{#1}}
%\newcommand{\mat}[1]{\mathbf{#1}} 

%\renewcommand{\vec}[1]{ \textbf{#1} }
%%%%%%%%%%%%%%%%%%%%%%

%%%%%%% THM, COR, DEF %%%%%%%
%\newtheorem{theorem}{Theorem}[section]
%\newtheorem{lemma}[theorem]{Lemma}
%\newtheorem{proposition}[theorem]{Proposition}
%\newtheorem{corollary}[theorem]{Corollary}
%\newenvironment{proof}[1][Proof]{\begin{trivlist}
%\item[\hskip \labelsep {\bfseries #1}]}{\end{trivlist}}
%\newenvironment{definition}[1][Definition]{\begin{trivlist}
%\item[\hskip \labelsep {\bfseries #1}]}{\end{trivlist}}
%\newenvironment{example}[1][Example]{\begin{trivlist}
%\item[\hskip \labelsep {\bfseries #1}]}{\end{trivlist}}
%\newenvironment{remark}[1][Remark]{\begin{trivlist}
%\item[\hskip \labelsep {\bfseries #1}]}{\end{trivlist}}
%\newcommand{\qed}{\nobreak \ifvmode \relax \else
%      \ifdim\lastskip<1.5em \hskip-\lastskip
%      \hskip1.5em plus0em minus0.5em \fi \nobreak
%      \vrule height0.75em width0.5em depth0.25em\fi}
%%%%%%%%%%%%%%%%%%%%%%

%
%\usepackage[algochapter]{algorithm2e}
%\usepackage[usenames]{color}
% colors to show the corrections
\newcommand{\red}[1]{\textbf{\textcolor{red}{#1}}}
\newcommand{\blue}[1]{\textbf{\textcolor{blue}{#1}}}
\newcommand{\cyan}[1]{\textbf{\textcolor{cyan}{#1}}}
\newcommand{\green}[1]{\textbf{\textcolor{green}{#1}}}
\newcommand{\magenta}[1]{\textbf{\textcolor{magenta}{#1}}}
\newcommand{\orange}[1]{\textbf{\textcolor{orange}{#1}}}
%%%%%%%%%% DK DK
% comments between authors
\newcommand{\toall}[1]{\textbf{\green{@@@ All: #1 @@@}}}
\newcommand{\toevan}[1]{\textbf{\red{*** Evan: #1 ***}}}
%\newcommand{\toevan}[1]{}  % USE FOR FINAL VERSION
\newcommand{\toe}[1]{\textbf{\red{*** Evan: #1 ***}}}
%\newcommand{\toe}[1]{\textbf{\red{*** Evan: #1 ***}}}
\newcommand{\tog}[1]{\textbf{\blue{*** Gordon: #1 ***}}}
%\newcommand{\togordon}[1]{\textbf{\blue{*** Gordon: #1 ***}}}

\renewcommand{\ge}[3]{{\textcolor{blue}{\strike{#1} #2}}\red{(#3)}}
%\renewcommand{\ge}[3]{{\textcolor{blue}{*** \textbf{Gordon:}\strike{#1} #2 ***}}\red{(#3)}}

\newcommand{\gea}[3]{{\textcolor{blue}{\textbf{(Accepted) Gordon:}\strike{#1} #2}}\red{(#3)}}
%\newcommand{\gea}[3]{{\textcolor{blue}{*** \textbf{(Accepted) Gordon:}\strike{#1} #2 ***}}\red{(#3)}}

\newcommand{\eb}[3]{{\textcolor{ForestGreen}{\strike{#1} #2}}\red{(#3)}}
%\newcommand{\eb}[3]{{\textcolor{ForestGreen}{*** \textbf{Evan:}\strike{#1} #2 ***}}\red{(#3)}}

%\def\ge#1#2#3{}{\textbf{\blue{*** Gordon: #2 ***}}}{(#3)}
\newcommand{\gee}[1]{{\bf{\blue{{\em #1}}}}}
\newcommand{\old}[1]{}
\newcommand{\del}[1]{***#1*** }



% \DeclareMathOperator{\Sample}{Sample}
%\let\vaccent=\v % rename builtin command \v{} to \vaccent{}
%\renewcommand{\vec}[1]{\ensuremath{\mathbf{#1}}} % for vectors
\newcommand{\gv}[1]{\ensuremath{\mbox{\boldmath$ #1 $}}} 
% for vectors of Greek letters
\newcommand{\uv}[1]{\ensuremath{\mathbf{\hat{#1}}}} % for unit vector
\newcommand{\abs}[1]{\left| #1 \right|} % for absolute value
\newcommand{\avg}[1]{\left< #1 \right>} % for average
\let\underdot=\d % rename builtin command \d{} to \underdot{}
\renewcommand{\d}[2]{\frac{d #1}{d #2}} % for derivatives
\newcommand{\dd}[2]{\frac{d^2 #1}{d #2^2}} % for double derivatives
\newcommand{\pd}[2]{\frac{\partial #1}{\partial #2}} 
% for partial derivatives
\newcommand{\pdd}[2]{\frac{\partial^2 #1}{\partial #2^2}} 
\newcommand{\pdda}[3]{\frac{\partial^2 #1}{\partial #2 \partial #3}} 
% for double partial derivatives
\newcommand{\pdc}[3]{\left( \frac{\partial #1}{\partial #2}
 \right)_{#3}} % for thermodynamic partial derivatives
\newcommand{\ket}[1]{\left| #1 \right>} % for Dirac bras
\newcommand{\bra}[1]{\left< #1 \right|} % for Dirac kets
\newcommand{\braket}[2]{\left< #1 \vphantom{#2} \right|
 \left. #2 \vphantom{#1} \right>} % for Dirac brackets
\newcommand{\matrixel}[3]{\left< #1 \vphantom{#2#3} \right|
 #2 \left| #3 \vphantom{#1#2} \right>} % for Dirac matrix elements
\newcommand{\grad}[1]{\gv{\nabla} #1} % for gradient
\let\divsymb=\div % rename builtin command \div to \divsymb
\renewcommand{\div}[1]{\gv{\nabla} \cdot #1} % for divergence
\newcommand{\curl}[1]{\gv{\nabla} \times #1} % for curl
\let\baraccent=\= % rename builtin command \= to \baraccent
\renewcommand{\=}[1]{\stackrel{#1}{=}} % for putting numbers above =
\newcommand{\diffop}[1]{\mathcal{L}#1}
\newcommand{\boundop}[1]{\mathcal{B}#1}
\newcommand{\rvec}[0]{{\bf r}}

\newcommand{\Interior}[0]{\Omega}
\newcommand{\domain}[0]{\Omega}
%\newcommand{\Boundary}[0]{\partial \Omega}
\newcommand{\Boundary}[0]{\Gamma}

\newcommand{\on}[1]{\hskip1.5em \textrm{ on } #1}

\newcommand{\gemm}{\texttt{GEMM}}
\newcommand{\trmm}{\texttt{TRMM}}
\newcommand{\gesvd}{\texttt{GESVD}}
\newcommand{\geqrf}{\texttt{GEQRF}}


\newcommand{\minitab}[2][l]{\begin{tabular}{#1}#2\end{tabular}}
\newcommand{\comm}[1]{\textcolor{red}{\textit{#1}}}

\newcommand{\nfrac}[2]{
\nicefrac{#1}{#2}
%\frac{#1}{#2}
}
 % color is defined in macros or misc_mac
% Rename  this file          misc_mac.tex
%----------------------------------------------------------------------
%%%%%%%%%%%%%%%%%%%%%%%%%%%%%%%%%%%%%%%%%%%%%%%%%%%%%%%%%%%%%%%%%%%%%%%%%%%%%%%
%
%	Math Symbols   Math Symbols   Math Symbols   Math Symbols   
%
%%%%%%%%%%%%%%%%%%%%%%%%%%%%%%%%%%%%%%%%%%%%%%%%%%%%%%%%%%%%%%%%%%%%%%%%%%%%%%%
\def\pmb#1{\setbox0=\hbox{$#1$}%
	\kern-.025em\copy0\kern-\wd0
	\kern.05em\copy0\kern-\wd0
	\kern-.025em\raise.0433em\box0}
\def\pmbf#1{\pmb#1}
\def\bfg#1{\pmb#1}

% BETTER VALUES FOR AUTOMATIC FIGURE PLACEMENT THAN THOSE PROVIDED BY 
% LATEX DEFAULTS.

\renewcommand{\textfloatsep}{1ex}
\renewcommand{\floatpagefraction}{0.9}
\renewcommand{\intextsep}{1ex}
\renewcommand{\topfraction}{.9}
\renewcommand{\bottomfraction}{.9}
\renewcommand{\textfraction}{.1}

% #1  position of floating figure (h|t|b|p)
% #1  EPS postscript file
% #2  size
% #3  caption

%usage of newfig:
%  \newfig{file.ps}{3in}{Fig1: this is a figure}

\input{epsf}
\def\newfig#1#2#3#4{
  \begin{figure}[htbp]
  \centering
  \vspace{1ex}
   \includegraphics[width=#2]{#1}
  %\setlength{\epsfxsize}{#2}
  \vspace{-.1in}\caption{\small #3}\break\vspace{.2in}
  \label{#4}
  \end{figure}
}

\def\herefig#1#2#3#4{
  \begin{figure}[h]
  \centering
  \vspace{1ex}
   \includegraphics[width=#2\textwidth]{#1}
  %\setlength{\epsfxsize}{#2}
  \vspace{-.1in}\caption{\small #3}\break\vspace{.2in}
  \label{#4}
  \end{figure}
}


%usage of newfigtwo: 2 figures, vertically stacked
% \newfig
%	{file1.ps}
%	{file2.ps}
%	{width}
%	{vertical space}
%	{Caption}

\def\newfigtwo#1#2#3#4#5{
  \begin{figure}[htbp]
  \vspace{1ex}
  \setlength{\epsfxsize}{#3}
  \centerline{\epsfbox{#1}}
  \vspace{#4}
  \setlength{\epsfxsize}{#3}
  \centerline{\epsfbox{#2}}
  \vspace{-.1in}\caption{\small #5}\break\vspace{.2in}
  \label{#1}
  \end{figure}
}

\def\newfigh#1#2#3#4{  % add height specification
  \begin{figure}[htbp]
  \vspace{1ex}
  \setlength{\epsfxsize}{#2}
  \setlength{\epsfysize}{#4}
  \centerline{\epsfbox{#1}}
  \vspace{-.1in}\caption{\small #3}\break\vspace{.2in}
  \label{#1}
  \end{figure}
}

\def\etal{{{\em et~al.\,\,}}}
\def\note#1{\\ =====#1===== \\}
\def\FBOX#1{\ \\ \fbox{\begin{minipage}{5in}#1\end{minipage}}\\ }
\newcount\sectionno     \sectionno=0
\newcount\eqnum         \eqnum=0
\def\addeqno{\global\advance \eqnum by  1 }
\def\subeqno{\global\advance \eqnum by -1 }
%\def\eqn{\addeqno \eqno \hbox{(\number\sectionno.\number\eqnum)} }

\def\tildetilde#1{\tilde{\tilde{#1}}}
\def\barbar#1{\overbar{\overbar{#1}}}

\def\vsp#1{\vspace{#1 ex}}
\def\fpar{\hspace{\parindent}}
%
%  \pf : 2 arguments: numerator and denominator of partial derivative
%
\def\pf#1#2{{\frac{\partial{#1}}{\partial{#2}}}}
\def\pfs#1#2{{\partial_{#2}{#1}}}
\def\pftwo#1#2{{\frac{\partial^2{#1}}{\partial{#2}^2}}}
\def\pfxx#1#2{{\frac{\partial^2{#1}}{\partial{#2}^2}}}
%\def\pfxy#1#2{{\frac{\partial^2{#1}}{\partial{#2}\partial{#3}}}}
\def\pfn#1#2#3{{\frac{\partial^{#1}{#2}}{\partial{#3}^{#1}}}}
\def\df#1#2{{\frac{d{#1}}{d{#2}}}}
\def\dfn#1#2#3{{\frac{d^{#1}{#2}}{d{#3}^{#1}}}}
\def\Dt#1#2{\frac{D#1}{D#2}}
\def\dt#1#2{\frac{d#1}{d#2}}
\def\bld#1{{\bf #1}}
\def\pfp#1#2#3{\pf{}{#3}{\left(\frac{#1}{#2}\right)}}

\def\norm#1{\|#1\|}

%
% Graphic characters  (\dot already defined by TeX/LateX)
%
\def\dash{\rule[1.5pt]{2mm}{.3mm}\HS{.9mm}}
\def\dott{\rule[1.5pt]{.7mm}{.3mm}\HS{.7mm}}
\def\dashline{\dash\dash\dash}
\def\dotline{\dott\dott\dott\dott\dott\dott}
\def\dashdotline{\dash$\cdot$\HS{.9mm}\dash}
\def\solidline{\rule[2pt]{7mm}{.3mm}}
% 
% overcircle
%
\def\ovcircle#1{\buildrel{\circ}\over{#1}}
%\def\below#1#2{\buildrel{#2}\under{#1}}
%\def\above#1#2{\buildrel{#2}\over{#1}}
%
%  big parenthesis and brackets
%
\def\bigpar#1#2{{\left(\frac{#1}{#2}\right)}}
\def\bigbra#1#2{{\left\[\frac{#1}{#2}\right\]}}

\def\Lp{\left(}
\def\Rp{\right)}
\def\Lb{\left[}
\def\Rb{\right]}
\def\Ln{\left\langle}
\def\Rn{\right\rangle}
\def\Ld{\left.}
\def\Rd{\right.}
\def\Lv{\left|}
\def\Rv{\right|}
\def\Lbr{\left|}
\def\Rbr{\right|}
\def\lng{\langle}
\def\rng{\rangle}
\def\Lc{\left\{}
\def\Rc{\right\}}
%%% %

% Cannot be handled by Lyx
%\def\[{{[}}
%\def\]{{]}}

%
\def\eol{\nonumber \\}
\def\eolnonb{\nonumber\\}
\def\eolnb{\\}
\def\nonb{\nonumber}
\def\be{\begin{equation}}
\def\ee{\end{equation}}
\def\BEQNA{\begin{eqnarray}}
\def\EEQNA{\end{eqnarray}}
\def\eqa{&=&}
\def\beqna{\begin{eqnarray}}
\def\eeqna{\end{eqnarray}}
\def\bverb{\begin{verbatim}}
\def\everb{\end{verbatim}}
\def\VERB#1{\bverb #1 \everb}
\def\btbl{\begin{tabular}}
\def\etbl{\end{tabular}}
\def\bmini{\begin{minipage}[t]{5.5in}}
\def\emini{\end{minipage}}
\def\parray#1#2{\left(\!\!\!\begin{array}{#1}#2\end{array}\!\!\!\right)}
\def\barray#1#2{\left[\begin{array}{#1}#2\end{array}\right]}
\def\carray#1#2{\left\{\begin{array}{#1}#2\end{array}\right.}
\def\darray#1#2{\left|\begin{array}{#1}#2\end{array}\right|}

\def\BEGTABLE#1{\begin{table}[hbt]\vspace{2ex}\begin{center}\bmini\centering\btbl{#1}}
\def\ENDTABLE#1#2{\etbl\caption[#1]{#2}\EMINI\end{center}\vspace{2ex}\end{table}}

\def\bfltbl#1{\begin{table}[hbt]\vspace{2ex}\begin{center}\bmini\centering\btbl{#1}}
\def\efltbl#1#2{\etbl\caption[#1]{#2}\emini\end{center}\vspace{2ex}\end{table}}
\def\mcol{\multicolumn}
%
%  label equations with (#)
%
\def\reff#1{(\ref{#1})}
%
%  macros borrowed from viewgraph package
%

\newenvironment{LETTRS}[3]{\begin{letter}{#1}
\input{origin}\opening{Dear #2:}\input{#3}\closing{Sincerely yours,}\end{letter}}{\clearpage}

\newenvironment{VIEW}[1]{{\BC\Huge\bf #1 \EC}\LARGE\VS{.05in}}{\clearpage}

\def\RM#1{\rm{#1\ }}
\def\BV{\begin{VIEW}}
\def\EV{\end{VIEW}}

\def\NI{\noindent}

\def\VS{\vspace*}
\def\HS{\hspace*}
\def\IT{\item}

\def\BARR{\begin{array}}
\def\EARR{\end{array}}

\def\BPARR{\left(\begin{array}}
\def\EPARR{\end{array}\right)}

\def\BDET{\left|\begin{array}}
\def\EDET{\end{array}\right|}

\def\BDF{\begin{definition}}
\def\EDF{\end{definition}}

\def\BSU{\begin{block}{Summary}}
\def\ESU{\end{block}}

\def\BEX{\begin{example}}
\def\EEX{\end{example}}

\def\BTH{\begin{theorem}}
\def\ETH{\end{theorem}}

\def\BCO{\begin{corollary}}
\def\ECO{\end{corollary}}

\def\BPROOF{\begin{proof}}
\def\EPROOF{\end{proof}}

\def\BLM{\begin{lemma}}
\def\ELM{\end{lemma}}

\def\BEQ{\begin{equation}}
\def\EEQ{\end{equation}}

\def\BEQNNB{$$}
\def\EEQNNB{$$}

\def\BE{\begin{enumerate}}
\def\EE{\end{enumerate}}

\def\BD{\begin{description}}
\def\ED{\end{description}}

\def\BI{\begin{itemize}}
\def\EI{\end{itemize}}

\def\BC{\begin{center}}
\def\EC{\end{center}}

\def\BFIG{\begin{figure}}
\def\EFIG{\end{figure}}

\def\BTABB{\begin{tabbing}}
\def\ETABB{\end{tabbing}}

\def\BMINI{\begin{minipage}}
\def\EMINI{\end{minipage}}

\def\BTABLE{\begin{table}}
\def\ETABLE{\end{table}}

\def\BTABUL{\begin{tabular}}
\def\ETABUL{\end{tabular}}

\def\MCOL{\multicolumn}
\def\UL{\underline}
\def\ULL#1{\UL{\UL{#1}}}

\def\BDOC{\begin{document}}
\def\EDOC{\end{document}}

\def\EM#1{{\em #1\/}}
\def\FN{\footnote}

% Courtesy of Ugo Piomelli

\def\latexfig #1 #2 #3 #4 #5 {\ \vfill
\hfill\hbox to 0.05in{\vbox to #3truein{
         \special{psfile="#1" angle=270 hscale=100 
                  hoffset=#4 voffset=#5 vscale=100} }\hfill}
\hfill\vspace{-0.1in}        }

% #1 is the .ps filename
% #2 is not used in the present version
% #3 is the size of the white space left above the caption (in inches)
% #4 is the horizontal offset from some unknown reference point.
%    It is in 1/72 of an inch and is positive to the right.
% #5 is the vertical offset from some unknown reference point.
%    It is in 1/72 of an inch and is positive upwards.


% Rename this file:    setupicase.tex
%----------------------------------------------------------------------
\setlength{\textwidth}{6.5in}
\setlength{\textheight}{9.0in}
\setlength{\topmargin}{-.1875pt}
\setlength{\oddsidemargin}{0pt}
\setlength{\evensidemargin}{0pt}
\setlength{\headsep}{0pt}
\setlength{\parskip}{1ex}
\setlength{\headheight}{0pt}



\def\red#1{\textbf{\textcolor{red}{#1}}}
\def\blue#1{\textbf{\textcolor{blue}{#1}}}
\def\qes#1{{\blue{*** For Erik: #1 ***}}}
\def\es#1{{\blue{*** For Erik: #1 ***}}}
\def\ge#1{{\red{*** For Gordon: #1 ***}}}
\def\ttt#1{{\tt #1}}
\def\bold#1{{\bf #1}}

\def\qes#1{}
\def\es#1{}
\def\ge#1{}
\usepackage{morefloats}


\usepackage{graphicx}
\usepackage[tight,footnotesize]{subfigure}
\usepackage{fixltx2e}
\usepackage{url}
\hyphenation{op-tical net-works semi-conduc-tor}


\begin{document}
\title{Sparse Matrix Vector Multiplication with Multiple vectors and Multiple Matrices on the
   MIC Architecture}


\author{\IEEEauthorblockN{Gordon Erlebacher\IEEEauthorrefmark{1},
Erik Saule\IEEEauthorrefmark{2}, Natasha Flyer\IEEEauthorrefmark{3}, 
and Evan Bollig\IEEEauthorrefmark{1}}
\IEEEauthorblockA{\IEEEauthorrefmark{1}Department of Scientific Computing, 
Florida State University, Tallahassee, FL 32306-4120\\
Email: gerlebacher@fsu.edu}
\IEEEauthorblockA{\IEEEauthorrefmark{2}Department of Computer Science, University of North Carolina at Charlotte\\
Email: erik.saule@uncc.edu}
\IEEEauthorblockA{\IEEEauthorrefmark{3}Computational and Information Systems Laboratory, UCAR \\
Email: 
flyer@ucar.edu}
\IEEEauthorblockA{\IEEEauthorrefmark{4}Department of Scientific Computing, Florida State University\\
Email: bollig@gmail.com}}
\maketitle


\begin{abstract}
In this paper, we develop an efficient scheme for the calculation of derivatives within the context
of Radial Basis Function Finite-Difference (RBFFD). RBF methods express functions as a linear
combination of radial basis functions on an arbitrary set of nodes. The Finite-Difference component
expresses this combination over a local set of nodes neighboring the point where the derivative is sought.
The derivative at all points takes the form of a sparse matrix/vector multiplication (spmv).

In this paper, we consider the case of local stencils with the number of nodes at each point and encode the
sparse matrix in ELLPACK format. We increase the number of operations relative to memory bandwidth by
calculating four derivatives of four different functions, or 16 different derivatives. We demonstrate
a novel implementation on the MIC architecture, taking into account its advanced swizzling and channel
interchange features. We present benchmarks that show an almost order of magnitude increase in speed
compared to efficient implementations of a single derivative. We explain the results through consideration
of operation count versus memory bandwidth.
\end{abstract}

\begin{IEEEkeywords}
OpenMP; MIC; spmv; sparse matrix; Radial Basis Function;
\end{IEEEkeywords}

\IEEEpeerreviewmaketitle



\section{Introduction}
\cite{Bell08, Vuduc05, Nishtala07, Stock12-TACC, Cuthill69, cramer2012openmp, Buluc2009_SPAA, Buluc11, Im01, Mellor-Crummey04, Nishtala07, Saad94sparskit, Williams07}
\cite{Temam:1992:CBS:147877.148091}
\cite{DBLP:journals/ijhpca/ShantharamCR11,
conf/ppsc/Toledo97, Liu:2013:ESM:2464996.2465013, Molka:2009:MPC:1636712.1637764,%
DBLP:journals/corr/abs-1101-0091, conf/ipps/BulucWOD11, conf/ipps/KreutzerHWFBB12,%
kumar2012accelerating, journals/concurrency/VazquezFG11}


SpMV is an important kernel for lots of stuff. So improving the
performance of SpMV has caputred the interest of many researcher.

The main challenge that is faced to improve good performance for SpMV
is that the operation are conducted using memory location that are
irregular and often unpredictible. That make that the kernels are
mostly memory bound and there is a significant instruction overhead
per flop.

Common improvement techniques such as register blocking, bandiwdth
reduction (matrix reordering), partitioning to fit in cache or TLB
have impacts which are very dependent on the matrix and overall do not
lead to dramatic improvement. Assuming register blocking does not
apply well to the matrix at hand (which is true for most matrices),
there is about 8 bytes of the matrix to move in per non zero (assuming
single precision); each nonzero requires two floating point operations
leading to a flop-to-byte ratio of at most $\frac{1}{4}$. This limits
the obtained performance to at most a quarter of the bandwidth of the
architecture wasting a lot of potentially useful cycles. The commonly
used techniques are mostly designed to reach that bound rather than
overcome it.

Fortunately that fate is not inevitable. One solution would be to pair
multiple component of an application to schedule a more instruction
intensive kernel simultaneously with SpMV, relying on some hardware
threading capabilities, such as HyperThreading, to reduce the cycle
wastage. However most ot the applications that use SpMV do not
typically have an instruction intensive kernel to run simultaneously.

An other solution, and the one we pursue in this paper, is to compute
multiple SpMV at once on matrices that have the same sparsity
patterns. Obviously not all the applications have such a property. But
important classes of applications such as graph
recommendation~\cite{}, eigensolving~\cite{} and the computation of
derivative of Radial Basis Functions(RBF)~\cite{} can use multiple
SpMVs simultaneously. In this paper in particular, we investigate the
case of the derivative of RBFs where four derivatives of four
different function is expressed as the multiplication of four vectors
by four matrices with identical sparsity patterns leading to the
simultaneous execution of 16 SpMVs at a time.

To perform our analysis, we focus our attention on the improvement
that we can achieve on the Intel Xeon Phi processor. It follow the
Many Integrated Core (MIC) architecture, which has a significant
memory bandwidth and peak flop throughput thanks to its 512-bit large
SIMD registers. The Xeon Phi processor has been shown to be promising
for sparse linear algebra compared to more classical CPU or GPU
architecture~\cite{}.

In Section~\ref{sec:rbf} we present the computation of RBFs and how it
can be expressed 16 multiplication of 4 vectors by 4 sparse matrices
with a common sparsity pattern. Section~\ref{sec:model} presents an
estimation of the instruction intensity of various form of the
computations and we show that a 5- to 6- fold improvement can be
expected when computing the 16 multiplications simultaneously and
reach a total of about 200 Gflop/s. This performance represents
approximatively 10\% of the available flop/s of a Knight Corner
coprocessor. Therefore, it is necessary to have implementations that
perform the computation in as little amount of instructions as
possible. We describe in Section~\ref{sec:impl} the details of the MIC
architecture and how to use specialized load, store, swizzle and
permutation instruction to efficiently bring the data in the vector
registers to be processed. Section~\ref{sec:expe} gives some
experimental result about the amount of bandwidth that can be achieved
depending on how the spmv kernel is written and the actual performance
of the various kernel on multiple classes of matrices some generated
for analysis purpose and some extracted from and RBF application. A
performance of xxx GFlop/s is achieved on real scenario. Concluding
remarks and perspectives are provided in Section~\ref{sec:ccl}.

\section{Derivatives of Radial Basis Functions}
\label{sec:rbf}


\section{Modelization of the Potential Improvements}
\label{sec:model}

\section{Efficient Implementation on the Intel Xeon Phi processor}
\label{sec:impl}

\section{Experimental Validation}
\label{sec:expe}

%\section{Problem Statement}
There has been much work on the problem of efficient sparse matrix/vector (spmv) and sparse matrix  (spmm) multiplication over the years \cite{} specialized for a range of computer architectures \cite{}. In the past 10 years, researchers have addressed implementations on the GPU using mostly CUDA \cite{}, and on multicore sytems, exemplified by the chips produced by Intel~\cite{}. In general, spvm consists of multiplying the matrix $A$ of size $N\times N$ by a vector $x$ of size $N\times 1$ to produce a vector $y$ of size $N\times 1$. More succinctly, $ y = A x $ .

In the theory of Radial Basis Functions Finite-Difference (RBFFD), derivatives of a function $f(\rvec)$ at node $i$ are expressed as a linear combination of the function values at the stencil center and the nodes connected to node $i$ (Figure~\ref{fig:rbf_stencils}). Thus $y$ is a discrete derivative of the vector $x$. 


When solving a one dimensional system of PDEs, one might require an $x$ derivative of multiple functions. For example, the Euler equations require the $x$ derivative of the three components of velocity and pressure. In this case, there are $n_v=4$ vectors $x^k$, $k=0,\cdots,3$. 

Thus for each vector element $y_i$, we compute $y_i = \sum_j A_{ij} x_j$. If $A_i$ is row $i$ of $A$, $y_i$ is simply the dot product $A_i x$. The next level of generality is to consider $n_v$ vectors $x^k$, $k=0,\cdots,n_v-1$. Whatever the spmv implementation, one achieves improved performance if the matrix formed from the columns $x^k$ are stored in row major order. Thus, $x^0_0,x^1_0,\cdots,x^{n_v-1}_0$, are stored in consecutive memory location. The random access of the elements of $x$ is thus reduced. Maximum efficiency is achieved when $n_v=16$ floats or $8$ doubles, given that cache lines take 64 bytes. We will benchmark this case, labeled $Svn$, where $n$ refers to the number of vectors (Iv4 uses four vectors). The $S$ refers to singe precision. A double precision run is labelled $Dvn$. 

Alternatively, when solving a PDE, one might require derivatives of a given scalar function with respect to coordinate directions $x$, $y$, $z$. Second order operators of often required, such as a second derivative with respect to $x$ or a Laplacian operator. In the RBFFD formulation, on can compute different derivatives using the same stencil, but with different weights. In other words, the adjacency matrix that corresponds to $A$ remains constant, but the matrix elements of $A$ change with the particular derivative. 
In this case, label with a superscript $l$ the particular matrix $A^k$. Since the adjacency matrix is assumed invariant, there is only need for a single matrix \ttt{C{ij}}. In Ellpack format, each row is of constant size (the number of nonzeros per row of $A$. $C_{ij}$ is the column number that locates the $j^{th}$ nonzero in row $i$ of $A$.

\ge{Must rewrite the above.}

%%% COPIED from Erik Saule paper
\section{The Intel Xeon Phi Coprocessor}
(USE FRODO AT MSI)
In this work, we use a pre-release KNC card SE10P. The card has 8 memory controllers where each of them can execute 5.5 billion transactions per second and has two 32-bit channels. That is the architecture can achieve a total bandwidth of 352GB/s aggregated across all the memory controllers. There are 61 cores clocked at 1.05GHz. The cores’ memory interface are 32-bit wide with two channels and the total bandwidth is 8.4GB/s per core. Thus, the cores should be able to consume 512.4GB/s at most. However, the band- width between the cores and the memory controllers is limited by the ring network which connects them and theoretically supports at most 220GB/s.

Each core in the architecture has a 32kB L1 data cache, a 32kB L1 instruction cache, and a 512kB L2 cache. The architecture of a core is based on the Pentium architecture: though its design has been updated to 64-bit. A core can hold 4 hardware contexts at any time. And at each clock cycle, instructions from a single thread are executed. Due to the hardware constraints and to overlap latency, a core never executes two instructions from the same hardware context consecutively. In other words, if a program only uses one thread, half of the clock cycles are wasted. Since there are 4 hardware contexts available, the instructions from a single thread are executed in-order. As in the Pentium architecture, a core has two di↵erent concur- rent instruction pipelines (called U-pipe and V-pipe) which allow the execution of two instructions per cycle. However, some instructions are not available on both pipelines: only one vector or floating point instruction can be executed at each cycle, but two ALU instructions can be executed in the same cycle.

Most of the performance of the architecture comes from the vector processing unit. Each of Intel Xeon Phi’s cores has 32⇥512-bit SIMD registers which can be used for double or single precision, that is, either as a vector of 8⇥64-bit values or as a vector of 16⇥32-bit values, respectively. The vector processing unit can perform many basic instructions, such as addition or division, and mathematical operations, such as sine and sqrt, allowing to reach 8 double precision operations per cycle (16 single precision). The unit also sup- ports Fused Multiply-Add (FMA) operations which are typically counted as two operations for benchmarking purposes. Therefore, the peak performance of the SE10P card is 1.0248 Tflop/s in double precision (2.0496 Tflop/s in single precision) and half without FMA.

%%\section{List of Tables and Figures}
%\begin{enumerate}
%\item
%Table: Notation for formulas
%\item
%Table: List of relevant vector instructions
%\item
%Figure: Matrix-Vector multiplication
%\item
%\item
%\item
%\item
%\end{enumerate}

\section{Suggested Benchmarks}
\begin{enumerate}
\item
SpMM with 16 floats (or 8 doubles) vectors (will give upper bound in speed). Implement algorithm  from Saul \etal. 
\item
SpMM with 4 floats or 4 doubles (to compare against SpMM). As a precursor to adding more matrices. 
\item
Try to figure out why our performance is so much lower than peak stated in Figure \ref{fig:gflops_peak_perf}.
Expecially in the case of compact. To do this, remove permutes, and replace gather by straight forward C++
without using vector notation, but using SIMD commands in addition to OPENPM commands. Make sure we use -O3, and contrast to -O1 and -O2. 
\end{enumerate}

\def\wide{2.5in}
\newfig{figures/rbf_stencils.pdf}{\wide}{RBFFD Stencils. Each node of the mesh is connected to $n_z-1$ stencil nodes in addition to itself. In the figure, node $A$ is connected to $B$, but $B$ is {\em not\/} connected to $A$. Thus adjacency graph of $A$ is not-symmetric. This must be corrected when reducing the bandwidth of $A$ with a Cuthill\-McKee algorithm, which assumes symmetry.}{fig:rbf_stencils}

\newfig{figures/matrix_structure.pdf}{\wide}{Matrix Structure.}{fig:mat_struct}
\newfig{figures/swizzling.pdf}{\wide}{Duplication of the first four bytes of each channel across the entire channel.}{fig:swizzling}
\newfig{figures/channel_permutation.pdf}{\wide}{Channel Permutations on the MIC}{fig:permutation}
\newfig{figures/tensor_product.pdf}{\wide}{Tensor scalar product using channels and swizzling.}{fig:tensor_product}

%\newfig{figures/theoretical_performance.png}{\wide}{Peak performance of SpMV algorithm assuming either 1 or 4 matrices and/or vectors. We used a memory bandwidth of 190 Gbytes/sec (based on best measurements with specialized code) and 150 Gbytes/sec (based on measurements in a code similar to the SpMV kernel, with computations removed.)}{fig:peak_perf}

\newfig{figures/gflops_peak.png}{2.5in}{Peak performance of SpMV algorithm assuming either 1 or 4 matrices and/or vectors. We used a memory bandwidth of 150 Gbytes/sec (based on measurements in a code similar to the SpMV kernel, with computations removed.)}{fig:gflops_peak_perf}

\newfig{figures/speedup_wrt_base.png}{\wide}{Speedup relative to base case using one matrix and one vector.}{fig:speedup}

\newfig{figures/test1_readwrite.png}{\wide}{Bandwidth performance under idealized conditions as a function of matrix row size. Entries with "cpp" denote cases where coding was performed without MIC vector instructions.}{fig:band_rw}

\newfig{figures/test1_gather.png}{\wide}{Bandwidth performance under idealized conditions as a function of matrix row size. Entries with "cpp" denote cases where coding was performed without MIC vector instructions.}{fig:band_gather}

\newfig{figures/test3_gather.png}{\wide}{Bandwidth performance under idealized conditions as a function of matrix row size. Entries with "cpp" denote cases where coding was performed without MIC vector instructions. The greater speed of the cpp version is obtained throught the use of \ttt{\#Ivdep} {\em and\/} \ttt{\_\_assumed\_aligned}. All memory is aligned on 64 bytes.}{fig:band_gather_ivdep}

\newfig{figures/host_test1_readwrite_no_temporal_hint.png}{\wide}{Bandwidth performance on the host under idealized conditions as a function of matrix row size. Entries with "cpp" denote cases where coding was performed without AVX vector instructions. The speed on the CPU matches the speed with AVX instructions. All memory is aligned on 32 bytes.}{fig:read_write}

\newfig{figures/mic_performance_nb_threads.png}{\wide}{Performance of $y=Ax$ on the MIC. Squares: base 1/1 case, solid circles: 4/4 case implemented in C++, solid triangles: 4/4 case implemented with MIC vector instructions. $-O3$ compilation options. Each group consists of four cases: grids of $64^3$ and $96^3$ with and without Reverse Cuthill McKee.}{fig:mic_performance}

% NOTES: why is result the same at 32 processors. I should find out the distribution of threads. 16 threads means one thread on each of 16 cores (YES, I BELIEVE), or 2 threads on each of 8 cores.} Must run the experiment with different nodes orderings. 
%export OMP_SCHEDULE=guided,64    (I should redo experiment with Dynamic or static and compact for KMP_AFFINITY
%export KMP_AFFINITY=compact
% MUST PRINT OUT OMP_SCHEDULE and KMP_AFFINITY inside the program. 
% WHY the very high performance with very few nodes? STRANGE. 
\newfig{figures/host_performance.png}{\wide}{Performance of $y=Ax$ on the Host. Squares: base 1/1 case, solid circles: 4/4 case implemented in C++, solid triangles: 4/4 case implemented with MIC vector instructions. $-O3$ compilation options. The four colors distinguish grid resolution and whether or not
Reverse Cuthill McKee is applied.}{fig:host_performance}

%\section{Bandwidth}
I measured bandwidth reduction using a Reverse Cuthill-McKee [], implemented 
in ViennaCL []. Two kinds of grids were considered. A 2D and a 3D Cartesian mesh of sizes ranging from $32^3$ to $64^3$ and $128^2$ to $256^2$. 
Stencils of size 32 and 64 were considered and computed using a Kd-Tree. 
(we also considered an overlayed grid (NO TIMINGS YET) and various forms of space filling curves. See Bollig thesis [].)  (There we no subdomain decompostion. In Bollig notation, the figures below correspond to a single 
square matrix. 

\begin{verbatim}
  N2D      reduced bandwidth  (stencil size: 32)
  128^2       1031
  256^2       2055    (bw = 8 * N)

        (stencil size 64)
  128^2      1522
  256^2      3059     (bw = 12 * N)


  N3D   reduced bandwidth (stencil size 32)
  32^3   4902
  64^3  11184    (bw = 4.8 * N^2)

  N3D   reduced bandwidth (stencil size 64) 
  32^3   6573   
  64^3   26923  (bw = 6.5 * N^2)
\end{verbatim}

Matrix representations after bandwidth reduction. (NEED FIGURE.)

\section{Register Density}
Calculate Register Density (RD) (Saule et al.). For each row of the \ttt{ col\_id} matrix, compute the number $n_c$ of cache lines touched by all the nonzero elements in the vector $x$ $(Ax)$. Divide the number of nonzeros $nnz$ by the number of elements that can be held $n_c$  cachelines, to obtained the 
$$
   RD = \frac{nnz}{n_e n_c}
$$
Each cache line is 64 bytes, which holds 16 floats or 8 doubles. 

We have computed the RD for a variety of 2D and 3D matrices with 32 or 64 nonzeros per row, that correspond to 2D ($32^2$, $64^2$, $128^2$ and $256^2$ grids, and 3D grids of size $32^3$, $64^3$, and $128^3$. The results are generated before and after bandwidth reduction (assume symmetric adjacency matrix (SHOW RESULTS: slightly better results than non-symmetric adjacency matrix (EXPLAIN IN PAPER.)

\begin{verbatim}
\end{verbatim}

%\section{Notation}
In this section, we introduce some variables that will prove useful to 
describe our experiments and construct a working model to examine the pros and cons of various assumptions. While deceptively simple, the MIC architecture requires a careful consideration of the properties of cache, memory bandwidth, cores and their interaction, threads per core (1-4), parallelization and vectorization. We will also seek to perform experiments that measure the best possible performance of the SPMV with maximum and minimum cache thrashing \qes{same as cache misses?}, with and without floating point operations, and with and without memory transfer. At the onset, we expect the dominant cost to e the transfer of the vector $x$ from memory, because contiguous elements of $x$ are not accessed sequentially. 

Relevant notation is contained in Table~\ref{tab:not}. 

\begin{center}
\begin{tabular}{|c|l|}
%\hline
%& & \\
\hline
$b_i$ & number of bytes per int \\
$b_x$ & number of bytes per float(4)/double(8) \\
$n_z$ & number of nonzeros per row $A$ \\
$n_r$ & number of rows of $A$ \\
$n_c$ & total number of elements in $col_id$: $n_z n_r$ \\
$b_w$ & matrix bandwidth \\
$b_c$ & total L2 cache per core ($512k=2^{19}$ bytes) \\
$b_T$ & theoretical minimum number of bytes transferred  \\
$b_{wT}$ & theoretical minimum number of bytes written to memory  \\
$b_{rT}$ & theoretical minimum number of bytes read from memory  \\
$n_C$ & total number of cores used  \\
$t_C$ & total number of threads used per core \\
$\rho$ & average cache density  \\
$n_v, n_m$ & number of vectors and matrices \\
$n_F$ & total number of floating point operations  \\
\hline
\end{tabular}
\end{center}

- weights and vector elements are either all floats or all doubles. 

Let us first estimate the memory (in a serial implementation) required to compute $y = Ax$ where $y$ has $n_r$ rows and $n_m$ columns, $A$ has
$n_c=n_r n_z$ nonzero elements, and $x$ is a vector of $n_r$ rows and 
$n_v$ columns. In this paper, we assume that the adjacency matrix for the
$n_m$ matrices $A$ is constant, but the values stored in the various $A$ differ. We are therefore executing $n_v n_m$ spmv operations. The objective of course, is to minimize wall-clock time on the MIC.

The total number of bytes is 
$$
   b_T = n_r (n_z b_i + n_z b_x n_m + b_x n_v)
$$
The ratio of $b_T$ to the total cache  over all cores is given by
$$
   R = \frac{b_T}{b_c}
$$
Clearly, the memory $b_T$ is composed of the memory required by the vectors $x$, $y$, the nonzeros of $A$ and the elements of the matrix \ttt{col}, which stores the nonzero columns in each row, and is an integral component of the Ellpack compressed matrix format specification. While the Ellpack format is less general than CSR, it is the idea structure for the RBBF simulations we are intersted in, wherein the number of RBF nodes per stencil is constant throughout the 2D or 3D grid and in time. 

Let us consider separately the number of bytes read from and written to memory.
Both the matrix $n_m$ matrices $A$ \qes{need an index on $A$ since there are more than one?} and \ttt{col\_id} are read from memory, as is $x$, whereas, the vector$y$ are stored to memory. In a later sections, we will discuss the practical performance of these operations and estimate the dominant contributions. 
\qes{We also wish to isolate the effects of read, write and compute operations as a function of the number of cores and threads to see how they influence the results and to estimate whether our results have the potential to scale similarly on systems with a higher number of cores. We "might" also compare are results against the best implementation using OpenCL, or do so in the paper.}

The total number of bytes read into memory is
$$ 
n_{rT} = n_r n_v b_x + n_r n_z b_i + n_r n_m n_z b_x
$$
while the total number of bytes written to memory is
$$
n_{wT} = n_r n_m n_v b_x
$$

Experiments similar to those performed in Saule \etal \cite{} suggest a maximum read  memory bandwidth of $190 Gbytes/sec$ using all threads on all four cores. The maximum speed advertised by Intel is $??? Gbytes/sec$, which we have found impossible to achieve in practice. To achieve $150 Gbytes/sec$ required allocating enough arrays to fill the 32 Gbytes of memory on the MIC, enabling prefetching, adding an explicit software prefetch, storing data as nrngo (prevent a rewrite to cache from memory prior to sending data to memory), and ensuring measurements in a steady-state regime.  With prefetching turned on, the maximum performances for each number of cores can be achieved with a single thread. However, when prefetching is turned off, performance increases as the number of active threads increases. Scaling is sublinear as a function of the number of cores (Fig. 1 in Saule \etal~\cite{}).  This suggests that when one computes the spmv, software prefetching might be useful to allow efficient transfer of data from memory {\em during\/} the calculation. 

Writing of data is less efficient with a maximum rate of 150 Gbytes/sec when using the No-Read hint and  no global reordering. \qes{We will return later to this issue. Prefetching with irregular access may not be practical}.
With only Vectorization and No-Read Hint, only 100 Gbytes/sec are achieved. Performance increase as the number of threads per core is increased, and the scaling is linear in the number of cores (Fig. 1b in \cite{Saule}). 

\qes{Perform your memory bandwidth experiments on my machine.}
\qes{How does Erik measure store and load separately?}

\subsection{Performance}
The best compute rate measured by Saule \etal~\cite{} is on the order of 20 Gflops for matrices 13 and 18, using $-O3$ compiler optimization \qes{icc compiler?} which have on the order of 100 and 200 average nonzero elements per row and a nonzero density on the order of $10^-3$. We work with matrices with density that ranges from $10^{-3}$ to $.3\times 10^{-5}$.  Lower density implies higher level of sparsity, and usually, higher sparsity leads to decreased performance \qes{may I say this?}. Based on Table 1, matrix 14 appears to have the closest characteristics to ours, with a constant number of nonzeros per row (41 nonzeros per row and a density of $10^{-5}$). From Figure 5 (Saule \etal), we see that matrix 22 has a cache line density between 0.15 and 0.25 . After bandwidth reduction, our matrices have a cache line density of about $0.15$, in single precision (I BELIEVE THIS WOULD double in double precision (not yet checked). All results in Saule \etal are in double precision \qes{can you confirm this?}. 

In the benchmarks to be described, we will demonstrate spmv at $100 Gflops$, or 5x the best performance in Saule \etal by simultaneously computing four spmvs of identical size where the four matrices have the same adjacency structure, with each spmv applied to four vectors. This would correspond in an RBFFD code to computing four derivatives of four functions, resulting in 16 vectors. 

\ge{I need a figure describing our three problems. Perhaps one figure for the 1x1 problem and one for the 4x4 problems.}
Rather than rely on the Intel compiler to perform the optimization, we program all our kernels in the Intel vector  syntax, combined with OpenMP pragmas. We will compare with straightforward implementations in C++. Naturally, the low level vector implementations cannot run on the host, or on previous generations of the Intel architecture. The advantage is that one gets a better feeling for the relationship between implementation and performance. If the Intel compiler is "good enough", better performance should be achieved, since, depending on the matrix parameters, more complex coding might be implemented, such as loop unrolling. 


\bold{Vector operations} \\
All matrices are aligned to 64 bytes, the size of a cache line, and the size of the vector registers. 
Each core has 32 vector registers, shared by 4 hardware threads. 
\qes{Is only a single thread responsible for loading up a vector register? So the other threads are either in a wait state or loading another register, or initiating a load/store?} \qes{I believe only two hardware threads can be involved in operations? That means 4 ops per second (using MADD) if both threads are active.}

Registers on the MIC are 512 bits wide, and can store 16 floats or 8 doubles. It is through these registers that the \#SIMD pragma is able to achieve vectorization. (In comparison to OpenCL or CUDA, one can think of these registers as a warp.) 
We use the following vector operations in our code, which we'll explain in the text. 

\def\loadps{\ttt{\_mm512\_load\_ps}}
\def\loadds{\ttt{\_mm512\_load\_ds}}
\def\fmadps{\ttt{\_mm512\_fmad\_ps}}
\def\gatherps{\ttt{\_mm512\_i32gather\_ps}}
\def\swizzleps{\ttt{\_mm512\_swizzle\_ps}}
\def\storenrngops{\ttt{\_mm512\_storenrngo\_ps}}
\def\castsi{\ttt{\_mm512\_castsi512\_ps}}
\def\permute{\ttt{\_mm512\_permute4f128\_epi32}}
\def\intmask{\ttt{\_mm512\_int2mask}}
\def\loadunpack{\ttt{\_mm512\_mask\_loadunpacklo\_epi32}}
\def\castitops{\ttt{\_mm512\_castsi512\_ps}}
\def\castpstoi{\ttt{\_mm512\_castps\_si512\_ps}}
%
\begin{center}
\begin{tabular}{|l|l|}
\hline
\loadps &  load 16 consecutive floats to register\\
\fmadps &  multiply/add of 16 floats\\
\gatherps &  gather 16, possibly disconnected floats from memory\\
\swizzleps &  reorder a channel, and duplicate across channels\\
\storenrngops &  store 16 floats to memory without register rewrite or reordering\\
\castpstoi & reinterpret 16 floats as integers\\
\castitops & reinterpret 16 integers as floats\\
\permute &  permute channels; do not change individual channels\\
\intmask &  \\
\loadunpack &  \\
\hline
\end{tabular}
\end{center}
%
In the vector instructions, \ttt{ps} refers to a float (4 bytes), while \ttt{epi32} refers to a 4-byte integer. Although we do not discuss double precision in this paper, \ttt{ds} refers to an 8 byte double precision real number (i.e., \loadds). 
Each vector register is broken up into four channels of 128 bits each. It is possible to interchange these channels at a cost of a "few" \qes{exact numbers?} cycles, and it is also possible to execute swizzle operations within a channel. For example, if the 128 bytes of each channel are labelled as $ABCD$, the vector instruction 
\ttt{\_mm512\_swizzle\_ps(v, \_MM\_SWIZ\_REG\_AAAA)}
replaces each channel by $AAAA$. 

Here is a function that reads in four floats (a,b,c,d) and creates the 16-float vector dddd,cccc,bbbb,aaaa: 
\begin{verbatim}
__m512 read_aaaa(float* a)                                                              
{
    int int_mask_lo = (1 << 0) + (1 << 4) + (1 << 8) + (1 << 12);
    __mmask16 mask_lo = _mm512_int2mask(int_mask_lo);
    __m512 v1_old;
    v1_old = _mm512_setzero_ps();
    v1_old = _mm512_mask_loadunpacklo_ps(v1_old, mask_lo, a);
    v1_old = _mm512_mask_loadunpackhi_ps(v1_old, mask_lo, a);
    v1_old = _mm512_swizzle_ps(v1_old, _MM_SWIZ_REG_AAAA);
    return v1_old;
}
\end{verbatim}

Here is a function that takes four floats from float register \ttt{v1}, and 
places them in each of the four lanes. Because the permulation function only 
operates on 4-byte integers, it is necessary to convert (in place) each float
to an integer (the bit structure is not changed), permute the lanes and cast
the integers back to floats (with no modification of the bit structure)
\begin{verbatim}
__m512 permute(__m512 v1, _MM_PERM_ENUM perm)                                           
{
    __m512i vi = _mm512_castps_si512(v1);
    vi = _mm512_permute4f128_epi32(vi, perm);
    v1 = _mm512_castsi512_ps(vi);
    return v1;
}
\end{verbatim}

%----------------------------------------------------------------------
\section{Base implementation}
We present an implementation written in C++ for the case of 4 matrices and 4 vectors to run on the host system and serve as base of comparison against more optimized code written with register based routines optimized for the MIC. Such an implementation is as follows:
\begin{verbatim}
\end{verbatim}
%----------------------------------------------------------------------
\section{Floating point operations}
We compute the floating point operations of the SpMV with general $n_v$ and $n_m$. Calling the total number of floating point operations $N_F$, we find
$$
  N_F = 2 n_r n_z n_v m_m
$$
Two cases must be distinguished in regards to $x$. In the first, in a worst case scenario, each element 
of $x$ is transferred $n_z$ times from memory. In the best case, the case of infinite cache, each element is transferred only once. Thus, we provide best and worst case scenarios and will compare against matrices $A$
in each of these extreme situations to evaluate the degree we approach idea performance. 
Thus the number of flops to bytes transferred is $N_F / (n_{rT} + n_{wT})$, which when written out, 
\begin{eqnarray}
N_{Fw} &=& \frac{2 n_r n_z n_v n_m}{b_x n_r n_z n_v  + n_r n_z b_i + n_r n_m n_z b_x + n_r n_m n_v b_x} \nonb \\
    &=& \frac{2 n_v n_m}{b_x ( n_v + n_m + n_m n_v n_z^{-1}) + b_i}  \nonb \\
%N_F &=& \frac{2 n_r n_z n_v n_m}{b_x n_r n_v (n_v+n_m) + n_r n_z b_i + n_r n_m n_z b_x} \nonb \\
    %&=& \frac{2 n_z n_v n_m}{b_x (n_v+n_m) + n_z (b_i + n_m b_x)} \nonb 
\end{eqnarray}
while in the best case, 
\begin{eqnarray}
N_{Fb} &=& \frac{2 n_r n_z n_v n_m}{b_x n_r n_v  + n_r n_z b_i + n_r n_m n_z b_x + n_r n_m n_v b_x} \nonb \\
    &=& \frac{2 n_v n_m}{b_x ( n_v n_z^{-1} + n_m  + n_m n_v n_z^{-1}) + b_i } 
\end{eqnarray}
In the special case $n_v=n_m=1$, 
$$
N_{Fw} = \frac{2}{b_x (2+n_z^{-1}) + b_i}
$$
and 
$$
N_{Fb} = \frac{2}{b_x (1 + 2 n_z^{-1}) + b_i}
$$
Neglecting $2$ compared to $n_z$, we find that $N_F = 2/(b_i+b_x)$, which leads to 
$N_F = 2/8=1/4$ and  $N_F=2/12=1/6$ for single and double precision, respectively. 
In the general case, in the condition of $n_z$ large compared to 2, 
$$
N_F =  \frac{2 n_v n_m}{n_m b_x + b_i}
$$ 
The case are interested in is $n_m=n_v=4$, so that
$$
N_F = \frac{8}{ b_x + 1}
$$
which is 8/5 and 8/9 in single and double precision, respectively. 
In single and double precision, we get $N_F=32/36=8/5$ and $N_F=32/68=8/9$, respectively.
The third column of the following table (TABLE ???) is the maximum achievable Gflop rate, assuming 
a maximum memory transfer speed of $150 Gflops$ (see earlier benchmark in this paper). 
(higher speeds can be achieved in practice,  upwards of 190 Gflops (Saule \etal\cite{}), but we use
a benchmark code similar to our current algorithm, without the indirect acessing required for $x$, but keeping the same use of vector registers used in our fastest implementation. (DO WE INCLUDE the CODE?)

\begin{center}
\begin{tabular}{|c|c|c|}
\hline
Precision & $n_m=n_v=1$ & $n_m=n_v=4$     \\
\hline
single    &  1/4  (37.5)     &   8/9 (133) \\
double    &  1/6  (25.5)     &   8/17 (62) \\
\hline
\end{tabular}
\end{center}
(With a 190 Gflop peak memory rate, maximum performance is 47.5 and 170 Mflops for the 1/1 case, while we calculate 32 and 90 Gflops in the 4/4/ case. 

Let us also calculate the flop to byte ratio for the case of 1/4 and 4/1 ($n_v$/$n_m$). One notices the symmetry with respect to $n_v$ and $n_m$. 
One gets $N_F = \frac{2*4}{5 b_x + 4}$ equal to 8/24 (=.33) and 8/44 (=.18), in single and double precision, respectively, which corresponds to 49 and 27 Gflops if a maximum of 150 Gflops is assumed.
As $n_m$ and $n_v$ keep increasing, $N_F$ tends towards $2/b_x$, equal to $0.5$ and $0.25$ respectively for single and double precision, which leads to peak speed of 75 and 37.5 Gflops.

In single precision, our best benchmarks achieve 20 Gflops for single precisions for on matrix/one vector, and 100 Glops for the 4/4 matrix/vector case, or 53\% and 75\% of peak performance, respectivey. We conclude that we achieve a higher percentage of peak performance for this algorithm when calculating 16 output $y$ vectors. The results are even more impressive when measured against the peak perforance of 2Tflops of the MIC in double precision.


\section{Conclusions and Perspectives}
\label{sec:ccl}

The conclusion goes here. this is more of the conclusion

\section*{Acknowledgment}


The authors would like to thank...
more thanks here


\bibliographystyle{plain} %{IEEEabrv,paper.bib}
\bibliography{paper, bollig}
\end{document}


