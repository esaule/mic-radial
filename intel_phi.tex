%% COPIED from Erik Saule paper
\section{The Intel Xeon Phi Coprocessor}
(USE FRODO AT MSI)
In this work, we use a pre-release KNC card SE10P. The card has 8 memory controllers where each of them can execute 5.5 billion transactions per second and has two 32-bit channels. That is the architecture can achieve a total bandwidth of 352GB/s aggregated across all the memory controllers. There are 61 cores clocked at 1.05GHz. The cores’ memory interface are 32-bit wide with two channels and the total bandwidth is 8.4GB/s per core. Thus, the cores should be able to consume 512.4GB/s at most. However, the band- width between the cores and the memory controllers is limited by the ring network which connects them and theoretically supports at most 220GB/s.

Each core in the architecture has a 32kB L1 data cache, a 32kB L1 instruction cache, and a 512kB L2 cache. The architecture of a core is based on the Pentium architecture: though its design has been updated to 64-bit. A core can hold 4 hardware contexts at any time. And at each clock cycle, instructions from a single thread are executed. Due to the hardware constraints and to overlap latency, a core never executes two instructions from the same hardware context consecutively. In other words, if a program only uses one thread, half of the clock cycles are wasted. Since there are 4 hardware contexts available, the instructions from a single thread are executed in-order. As in the Pentium architecture, a core has two di↵erent concur- rent instruction pipelines (called U-pipe and V-pipe) which allow the execution of two instructions per cycle. However, some instructions are not available on both pipelines: only one vector or floating point instruction can be executed at each cycle, but two ALU instructions can be executed in the same cycle.

Most of the performance of the architecture comes from the vector processing unit. Each of Intel Xeon Phi’s cores has 32⇥512-bit SIMD registers which can be used for double or single precision, that is, either as a vector of 8⇥64-bit values or as a vector of 16⇥32-bit values, respectively. The vector processing unit can perform many basic instructions, such as addition or division, and mathematical operations, such as sine and sqrt, allowing to reach 8 double precision operations per cycle (16 single precision). The unit also sup- ports Fused Multiply-Add (FMA) operations which are typically counted as two operations for benchmarking purposes. Therefore, the peak performance of the SE10P card is 1.0248 Tflop/s in double precision (2.0496 Tflop/s in single precision) and half without FMA.
